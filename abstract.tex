\begin{abstract}

Generative neural networks offer powerful tools for the generation of data in many domains, given their ability to model distributions of data and generate high-fidelity results. 
However, a major shortcoming is that they are unable to explicitly diverge from the training data in creative ways and are limited to fitting the target data distribution.
This thesis presents a body of work investigating ways of training, fine-tuning, and configuring generative neural networks in inference in order to achieve data-divergent generation.
This goal of configuring generative neural networks to diverge from their original training data or any existing data distribution is referred to as \textit{active divergence}.
All of the approaches presented in this thesis are data-free in their implementation, which inherently distinguishes these approaches from the traditional orthodoxies of imitation-based learning that is widespread throughout most machine learning research. 
The research presented in this thesis represents three categorical contributions to achieving active divergence: training without data, divergent fine-tuning, and network bending.
In addition to this, a formal survey and taxonomy of active divergence methods is presented as another contribution of this thesis. 
The overriding goal of the research in thesis is to \textit{expand the generative space} of generative neural networks. 
All three methods presented achieve this, and point to a new approach to working with generative AI that does not rely on the imitation of, and derivation from data, for extracting its value and creative possibilities.


\end{abstract}
