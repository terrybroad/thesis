\chapter{Discussion}
\label{ch:discussion}

In this chapter, I will reflect on the work done in this thesis, and situate it in the context of other developments in the trends that have occurred in CreativeAI and AI-Art that have taken place during the course of work on this thesis. 
Many of the arguments presented here were also disseminated in the paper `Using Generative AI as an Artistic Material: A Hacker's Guide’ that I published at the 2nd international workshop on eXplainable AI for the Arts (xAIxArts) at the ACM Creativity and Cognition Conference \citep{broad2024using}.


\section{Hacking as Research Methodology}

Hacking has many definitions that encompass technical practices, subcultures and ethical philosophies \citep{jordan2017genealogy}. 
Whilst it often gets associated with jailbreaking and circumventing cybersecurity measures, what \citep{stallman2002hacking} labels as \textit{cracking}.
Hacking encompasses a much broader approach to working with technology.
Eryk Salvaggio, one of the members of the Algorithmic Resistance Research Group (ARRG!), labels their approach to understanding complex algorithmic structures through artistic practices as `the creative misuse of technology' \citep{salvaggio2023cultural}.

Hacking is also understood through practices of making, design and technological experimentation \citep{hunsinger2016democratization}.
Here, hackers are defined as those `who are interested in acquiring knowledge about programming systems by venturing beyond their limits' and are understood as `skilled individuals who possess proficiency in network and computer systems as well as a desire for intellectual challenges' \citep{richterich2017introduction}.
The widespread phenomenon of hackathons, where technical practitioners are invited to work intensively, usually around a specific technology and develop a new invention by playfully using that technology has become seen as an approach to design research through making \citep{flus2021design, falk2022future, rys2023invention}.

Hacking can also be viewed as a performative act. In `Hacking Perl in nightclubs', \cite{mclean2004hacking}, describes a musical artistic practice of coding live music in nightclubs, where the playful experimentation of code is the live artistic practice itself. 
This practice has spawned an entire discipline of creative practice and academic research of live coding \citep{selvaraj2021live}.
In live coding, coding itself is the creative outlet, which is explored in a performative setting, where code itself is understood as both aesthetic and political expression \citep{cox2012speaking}.

Electronic hardware is something that can be tinkered with, experimented with and hacked \citep{collins2004hardware, grand2004hardware}. 
In hardware hacking, repurposing existing electronic hardware can be used for music-making \citep{collins2009handmade}, and other forms of physical expressions \citep{hartmann2008hacking}.
Exploring the limits of specialist technical equipment, and repurposing it for unintended acts allows for new, divergent possibilities in design to be achieved through playful experimentation \citep{goddard2015playful}.

Just like hardware electronics, I see generative neural networks as complex structures, with many contigencies made up of discrete, tinkerable elements.
Throughout the research conducted in this thesis, this approach of hacking deliberately tries to break the normal functions of generative neural networks during inference and training. 
All three of the chapters of original research in this thesis came from deliberately trying to break these models.
The line of enquiry actively goes against the common assumptions of orthodoxies of AI research, which is both heavily formalised and driven by ideology \citep{sias2021ideology} that conforms to techno-optimist \citep{andreesen2023techno} and technological determinism \citep{drew2016technological} philosophies. 

My goal with the research presented in this thesis was to find an alternative way of working with and thinking about generative AI that escapes narratives of technological determinism, and decentering of human agency in creative practice by AI \citep{zeilinger2021tactical}.
All of the outcomes of the research experiments are technological interventions, that produce aesthetic outcomes that help us better understand the functioning of technological systems used to produce it (\S \ref{c8:sec:explaining}). 
I would not have been able to come up with the ideas that I did for all of the methods introduced, were it not for this approach of hacking that I took for the development of all of these works.

Each of the experiments resulted in artwork.
It was the technological intervention itself (aka the hack) that was the key to determining the aesthetic outcomes in these works and building a narrative for viewers to understand them.
The generated outputs are the means through which we can better understand these complex technical systems by repurposing them.
In all of these experiments, I view \textit{the hack} as where the creative agency has occurred, and I do not consider any of these works as having shared creative agency with the algorithms \citep{moruzzi2022creative}, which is a more common framing in the discourse around AI-art.
Instead of viewing AI algorithms as tools, or means to automate human agency out of the creative process, I have viewed these systems as artistic materials in their own right (\S \ref{c8:sec:material}).

\section{The Role of Aesthetic Judgement in Generative AI Research}
\label{c8:sec:aesthetic}

Aesthetic judgment is a commonly used metric in determining the progress of research in AI \citep{stanley2018art}, even if this is not explicitly stated in the values used to measure the progress of AI research \citep{birhane2022values}.
The aesthetic values that drive generative AI research are often those of realism, perfect imitation, and creating outputs indistinguishable from human outputs (as was first formulated in the imitation game \citep{machinery1950computing}). 

In the research presented in this thesis, aesthetic judgement has been central to the development of and evaluation of the experiments and generated outputs, but here the guiding aesthetic qualities have been novelty and divergence from the qualities of human outputs in the original training datasets.
Whilst aesthetic judgment is an imprecise measure, the `effectiveness of various computational media processes in improving creative output is the most substantial measure of their value' \citep{brown2009integrating}. 
In my research, the effectiveness has been evaluated as how novel the aesthetic output is. 
In determining whether an intervention would qualify as being novel, I have used the yardstick of how easily the generated outputs would be able to be reproduced using other, more conventional methods.
If the outputs are so distinct as to be impossible to reproduce by other means (including more conventional approaches to using generative AI), that has been the means by which I have considered various experiments as having enough value to the wider world that it should be disseminated through academic publishing and using the outputs of these experiments as artworks in their own right (which I have discussed in extensive detail in Chapter \ref{ch:impact}).

\section{Using Generative AI as an Artistic Material}
\label{c8:sec:material}
Throughout all three of the chapters of the original work presented in this thesis, I took the approach of taking generative AI models, and the code that is used to train models as artistic materials themselves.
Taking a hacking approach, artists can use AI in non-normative ways to create new methods of working with AI that both reveal the inner workings of AI (\S \ref{c8:sec:explaining}) and produce new routes for artistic expression. 
In \cite{broad2024using}, I classify this into four approaches: \textit{subverting a network's inputs}, \textit{upending a network's training}, \textit{corrupting a network's weights}, and \textit{hacking the computational graph}. 
The original work presented in this thesis falls into two of these categories: upending a network training and hacking the computational graph.

Artists projects like Phillip Schmitt's \textit{Introspections} \citep{schmitt2019introspections} and Eryk Salvaggio's \textit{Writing noise into noise} \citep{salvaggio2023noise} are examples of \textit{subverting a networks inputs}. 
In \textit{Introspections} \cite{schmitt2019introspections}, the artist Philipp Schmitt took off-the-shelf image translation models, designed to translate photographs into line drawings and vice-versa and fed into them blank images. At first, the images returned were themselves blank, but after the outputs were repeatedly fed back into the same model many times, detailed artefacts emerged, showing complex hallucinations from the model's internal operations.
In \textit{Writing noise into noise}, Salvaggio prompted denoising diffusion models \cite{sohl2015deep} to generate images of \textit{`Gaussian noise'}, something that they are ironically very bad at doing.

Mario Klingemann's \textit{Neural glitch} (also discussed in \S \ref{c2:subsec:divergent-practice}; \S \ref{survey:rewriting}) is an example of \textit{corrupting the weights of a network}. 
Through the processes of altering and corrupting the learned parameters of the network, Klingemann helps to reveal its inner functionality.

Chapters \ref{ch:unstable_eq} \& \ref{ch:divergent} both represent approaches to \textit{upending a network's training}. 
The experiments in Chapter \ref{ch:unstable_eq} are the most explicit way of approaching the code frameworks of a generative neural network as an artistic material.
In this work, I view the approach as akin to practices in traditional generative art, where dynamic systems are built and the role of the artist is to design or influence this process to some degree, based on intuition and exploration \cite{mccormack2004generative}.
I consider this work to be in the category of artistic practice described by Bense as \textit{Generative Aesthetics}.
The only difference is that instead of using deterministic computer code, I have used the modern tools of GPU-optimised linear algebra libraries, differentiable objective functions and gradient-based optimisation to design and explore the characteristics of these new `aesthetic structures' \cite{bense1965projekte}.

Chapter \ref{ch:divergent} more explicitly applies the approach to hacking in subverting the normal functioning of pre-existing loss functions (\cite{berns2020bridging} explicitly labels this approach as \textit{loss hacking}).
By inverting the adversarial loss, I was able to both reveal an otherwise unseen aspect of the discriminator's hidden perception (which is crucial to effectiveness in the fidelity of GANs) and create an explicit approach to actively diverging from data.

The original working title for the Network Bending paper (detailed in Chapter \ref{ch:net_bend}) was \textit{hacking the computational graph}.
The computational graph is the term given for the chain of computations, as defined by the input data, learned parameters, network topology, and computational functions that define the forward pass of a neural network (aka inference). Network bending allows for interventions into the computational flow of a model during inference. 
This approach allowed for a flexible and direct way of artistic manipulation of the internal representation of a generative model, using deterministically controlled filters that are inserted as their own layers into a generative model.
The goal of this being was to allow artists a direct and expressive mechanism over the flow of computation within the models themselves.

The breadth of artworks made with network bending \S \ref{c7:sec:net-bend-artworks} and ways that network bending has been extended \S \ref{c7:sec:net-bend-impact} shows the flexibility and appeal that this approach has had.
It is clear that many artists want to intervene in the generative processes afforded by generative neural networks, not simply to regurgitate existing data, but intervene in the computational processes underlying it.

\section{Explaining AI through Artistic Enquiry}
\label{c8:sec:explaining}

Generative neural networks produce media through a complex fabric of computation, contingent on large scraped datasets, where features and representations get encoded into the weights of unfathomably large data arrays, which in turn are enmeshed through complex chains of computation. 
The ease and realism through which this generated media is mass-produced and its almost uncanny flawlessness \cite{smith2023ai} makes it easy to forget the complex computational contingencies that produce it. 
I argue that the work presented in this thesis shows that rather than simply using generative neural networks as a tool, treating it critically as an artistic material can help bring this complex fabric of computation to the fore. 

These approaches are not dissimilar to the \textit{glitch art} and \textit{databending} movements that were likewise seeking to reveal, through imperfection, otherwise hidden aspects and material functionality of digital media \cite{kemper2023glitch}.
Making targeted interventions into to inputs, weights, training and inference of generative neural networks, artists are able to make critical works that reveal to us otherwise unseen aspects of these models generation. 
Taking a hacker's ethos to generative neural networks provides a critical approach for explainable AI (XAI) in the arts.
Where the artworks themselves present new ways of understanding and making sense of these unfathomably complex computational systems. 

\section{Assessing Impact Through Generalisation}
\label{c8:sec:generalisation}

An important measure of impact, when it comes to practice-led interventions and hacks in creative media technologies is generalisation.
Being able to reuse, and reapply an intervention `is important not only for making contributions to society at large through effective knowledge transfer but also to empower the researcher/practitioner in their future work' \citep{brown2009integrating}.
If we assess various chapters of original research, presented in this thesis (Ch.\ref{ch:unstable_eq}, \ref{ch:divergent} \& \ref{ch:net_bend}). 
Then the network bending framework (Ch.\ref{ch:net_bend}) is clearly the original research contribution that has had the most impact. 
Network bending has been widely used by both artists (\S \ref{c7:sec:net-bend-artworks}) and extended into new generative domains and paradigms for interactions by other researchers (\S \ref{c7:sec:net-bend-impact}).
The generalisability of network bending as an approach to both hacking and intervening in models, and in increasing the creative agency of artists over the functioning of generative neural networks.
In addition to this, network bending is the most flexible way of \textit{expanding the generative space} of AI models, beyond the straightforward imitation of data, and towards the repurposing of these models towards new aesthetic possibilities. 
